\newcommand{\decktitle}{Einführung}
%%%%%%%%%%%%%%%%%%%%%%%%%%%%%%%%%%%%%%%%%%%%%%%%%
%
% DOCUMENT
%
%%%%%%%%%%%%%%%%%%%%%%%%%%%%%%%%%%%%%%%%%%%%%%%%%

\begin{frame}
    \subtitle{\decktitle}
    \titlepage
\end{frame}


\begin{frame}
    \frametitle{\textbf{Outline:}}
    \tableofcontents
\end{frame}


		
  
\section{Vorstellung}  
    \begin{frame}{Über mich}
        \begin{columns}
            \begin{column}{0.3\textwidth}
                \includegraphics[width=\textwidth]{chapters/01_introduction/figures/me.png}
            \end{column}
            \begin{column}{0.6\textwidth}
                \begin{itemize}
                    \item Jonas Miederer
                	\item 27 Jahre
                	\item Hauptberuflich tätig bei der Daimler AG als Softwareentwickler für Künstliche Intelligenz
                	\item Studium an der Hochschule der Medien (Bachelor: Medieninformatik, Master: Computer Science And Media)
                	\item Bei Fragen, Problemen oder Unklarheiten gerne melden: \href{mailto:dhbw@jonas-miederer.de}{hello@jonas-miederer.de} oder \href{https://jonas-miederer.de/}{https://jonas-miederer.de/}
          	    \end{itemize}
            \end{column}
        \end{columns}
    \end{frame}
          
   \section{Über die Veranstaltung}  
     \begin{frame}{Inhalte}
        
        \begin{itemize}
            \item Grundlagen der Informatik
            \item Kernanwendungen und aktuelle Themen der Informatik
            \item Algorithmen, Programm- und Datenstrukturen
            \item Problemlösung mit modernen Programmier-/Skriptsprachen
            \item Datenbankverwaltung, -entwurf \& -implementierung
            \item Datenbankkonzepte
        \end{itemize}
    \end{frame}
    
    \begin{frame}{Ziele}
        \begin{itemize}
            \item Grundverständnis der Konzepte in der Informatik
            \item Lösungskompetenz unterschiedlicher Problemstellungen in der Informatik
            \item Implementierung eigener Algorithmen und Funktionalitäten
        \end{itemize}
    \end{frame}
    
    \begin{frame}{Termine}
        \textbf{Semester 1:}
        \begin{itemize}
            \item 27.11.2020
            \item 04.12.2020
            \item 11.12.2020
            \item 18.12.2020
            \item 08.01.2021
            \item 15.01.2021
            \item 22.01.2021
        \end{itemize}
        
        \textbf{Semester 2:}
        \begin{itemize}
            \item 14.05.2021 (Prüfungsvorbereitung)
        \end{itemize}
    \end{frame}
    
    \begin{frame}{Organisation \& Prüfungsleistung}
    
        \begin{itemize}
            \item Kombination aus Theorie \& Praxis
            \item Übungen zur Anwendung des Gelernten
            \item Workload 150 Stunden (74 Präsenz / 76 Selbststudium) \newline
            \item Prüfungsleistung: Programmentwurf
            \item 5 ECTS-Punkte
        \end{itemize}
     \end{frame}
    
