\newcommand{\decktitle}{Python IV - Weitere Themen}

%%%%%%%%%%%%%%%%%%%%%%%%%%%%%%%%%%%%%%%%%%%%%%%%%
%
% DOCUMENT
%
%%%%%%%%%%%%%%%%%%%%%%%%%%%%%%%%%%%%%%%%%%%%%%%%%

\begin{frame}
    \subtitle{\decktitle}
    \titlepage
\end{frame}


\begin{frame}
    \frametitle{\textbf{Outline:}}
    \tableofcontents
\end{frame}

		
\section{Exceptions}

    \begin{frame}{Exceptions (I)}
        Manchmal treten bei der Programmierung unvorhergesehene Fehler auf. Dies passiert häufig während der Entwicklung eines Programms und kann somit einfach behoben werden. Befindet sich eine Software jedoch im Produktivbetrieb, kann ein unvorhergesehener Fehler unter Umständen zu weitreichenden und schwerwiegenden Konsequenzen führen. \\~\
        
        Daher sollte bereits während der Planung und Entwicklung eines Programms auf die Qualität und Fehlerfreiheit geachtet werden.
    \end{frame}
    
    \begin{frame}{Exceptions (II)}
        Sollte während der Ausführung eines Programms ein Fehler auftreten, so wird vom Python Interpreter automatisch eine sogenannte \textbf{Exception} erzeugt. \\~\
        
        Der Name \textit{Exception} kommt daher, dass im Falle eines Fehlers eine Ausnahmebehandlung durchgeführt werden kann. Das bedeutet, dass der Fehler beispielsweise im weiteren Verlauf des Programms behoben werden kann (siehe Exeption-Handling). \\~\
        
        Ohne das sogenannte "Abfangen" einer Exception führt eine Exception grundsätzlich zum Abbruch des Programms. Der Grund hierfür ist, dass der Python Interpreter nicht weiß, wie mit einem möglichen Fehler umgegangen werden soll und da sich das Programm nicht in einem undefinierten Zustand befinden kann, wird es beendet.
    \end{frame}
    
    \begin{frame}[fragile]{Exceptions (III)}
        Sollte aus irgendeinem Grund eine Exception erzeugt werden, wird mithilfe des sogenannten \textbf{Stack-Traces} angezeigt
        \begin{enumerate}
            \item was der Grund des Fehlers ist
            \item wo der Fehler aufgetreten ist
            \item was den Fehler ausgelöst hat
        \end{enumerate}
        
        \\~\
        
        Dies kann am einfachsten durch die folgenden Beispiele dargestellt werden:
        
    \end{frame}
    
    \begin{frame}[fragile]{Exceptions (IV)}
\begin{exampleblock}{Beispiel 1: Division durch 0}
\begin{pyconcode}
>>> 1/0
Traceback (most recent call last):
  File "<stdin>", line 1, in <module>
ZeroDivisionError: division by zero
\end{pyconcode}            

Hier wurde der Fehler durch eine Division mit 0 ausgelöst. Daher wird ein \textbf{ZeroDivisionError} erzeugt ("geworfen").
\end{exampleblock}
    \end{frame}
    
    \begin{frame}[fragile,allowframebreaks]{Exceptions (V)}
\begin{exampleblock}{Beispiel 2: Datei nicht gefunden}
\begin{pythoncode}
def open_file(filename):
    open(filename)


if __name__ == "__main__":
    open_file("test.txt")
\end{pythoncode}
\begin{pyconcode}
Traceback (most recent call last):
  File "exception.py", line 6, in <module>
    open_file("test.txt")
  File "exception.py", line 2, in open_file
    open(filename)
FileNotFoundError: [Errno 2] No such file or directory: 'test.txt'
\end{pyconcode}            

\end{exampleblock}
\begin{exampleblock}

Hier wurde der Fehler ausgelöst, indem eine Datei namens "test.txt" geöffnet werden soll, die aber im aktuellen Verzeichnis nicht existiert. Daher wird ein \textbf{FileNotFoundError} geworfen. \\~\

Wird der Code direkt über die Kommandozeile eingegeben (wie im 1. Beispiel), tritt der Fehler unmittelbar nach der Eingabe bzw. Ausführung der problematischen Anweisung auf und kann daher leicht lokalisiert werden. Wird der Code jedoch über den Inhalt einer oder mehrerer Dateien ausgeführt, die mitunter hunderte Zeilen besitzen können, fällt es schwerer, den Fehler genau zu lokalisieren, da der Fehler erst bei einer Ausführung des gesamten Programms auftritt und nicht direkt bei der Eingabe des jeweiligen Befehls. \\~\

Daher bietet der Stack-Trace zusätzliche Informationen, wo genau der Fehler aufgetreten ist. Im obigen Beispiel ist etwa zu erkennen, dass der Fehler in Zeile 2 der Datei \code{exception.py} aufgetreten ist. Dieser Aufruf wurde wiederum durch einen Aufruf in Zeile 6 der selbigen Datei ausgelöst. Mithilfe des Stack-Traces kann also genau zurückverfolgt werden, wo der Fehler auftritt und welche Funktionsaufrufe diesen ausgelöst haben.
\end{exampleblock}
    
    \end{frame}
    
    \begin{frame}[fragile,allowframebreaks]{Exceptions (VI)}
        
        \begin{exampleblock}{Beispiel 3: String-/Integer Konkatenation}
\begin{pythoncode}
def get_user_info():
	name = input("Wie heißt du?: ")
	age = int(input("Wie alt bist du?: "))

	return name, age

def print_user_info():
	name, age = get_user_info()
	print("Hallo " + name + ", du bist " + age + " Jahre alt." )

if __name__ == "__main__":
	print_user_info()
\end{pythoncode}
\end{exampleblock}

\vbox{
\begin{exampleblock}{Beispiel 3: String-/Integer Konkatenation}
\begin{pyconcode}
"Wie heißt du?: Jonas"
"Wie alt bist du?: 27"
Traceback (most recent call last):
  File "test.py", line 12, in <module>
    print_user_info()
  File "test.py", line 9, in print_user_info
    print("Hallo " + name + ", du bist " + age + " Jahre alt." )
TypeError: can only concatenate str (not "int") to str
\end{pyconcode}            
\end{exampleblock}
}

    \end{frame}
    
    \begin{frame}[fragile]{Exception Handling (I)}
        Da im Falle einer auftretenden Exception die Ausführung des Programms ohne weitere Maßnahmen des Programmierers grundsätzlich abgebrochen wird, stellen diese Fehler häufig ein unerwünschtes Verhalten dar, da die Software im Produktivbetrieb möglichst fehlerfrei und stabil laufen sollte.\\~\ 
        Dennoch können auftretende Exceptions meist nicht vollständig vermieden werden, insbesondere bei der Verarbeitung von Daten, die während der Entwicklung noch nicht bekannt sind, sondern erst während der Ausführung des Programms eingelesen werden (z.B. durch Nutzereingaben, Datenbankabfragen, ...).\\~\
        Da diese Daten zunächst nicht bekannt sind, muss sich der Entwickler überlegen, welche Fehler bei der späteren Verarbeitung der Daten potentiell auftreten könnten, z.B. Speicherort der Daten existiert nicht, Umwandlung von Datentypen schlägt fehl, mathematisch ungültige Rechenoperationen, etc.
    \end{frame}
    
    \begin{frame}[fragile]{Exception Handling (II)}
        Der Programmierer muss sich bereits während der Entwicklung des Programms um diese möglicherweise auftretenden Fehler kümmern, sodass diese nicht unmittelbar zu einem Absturz des Programms führen. Falls also eine Exception auftreten sollte, muss der Programmierer definieren, was stattdessen passieren soll. Dieses Vorgehen wird als \textbf{Exception Handling} bezeichnet.
    \end{frame}
    
    \begin{frame}[fragile]{Exception Handling (III)}
        Das Exception Handling besteht immmer aus mindestens 2 Teilen:
        \begin{enumerate}
            \item Zunächst wird angegeben, welche Funktion das Programm ausführen soll. Dies ist die Stelle, an der die Exception auftreten kann (\textbf{try}-Block)
            \item Code, der ausgeführt wird, falls die Exception auftritt. Da sich das Programm nicht in einem undefinierten Zustand befinden kann, muss der Programmierer angeben, was stattdessen passieren soll (\textbf{except}-Block)
        \end{enumerate}\\~\
        
        Das Exception Handling wird immer mit den Schlüsselwörtern \code{try} und \code{except} beschrieben.
    \end{frame}
    
    \begin{frame}[fragile]{Exception Handling (IV)}

    
\begin{exampleblock}{Beispiel: Division}
In folgendem Beispiel werden 2 Zahlen vom Nutzer eingelesen, dividiert und das Ergebnis schließlich ausgegeben.

\begin{pythoncode}
num_1 = float(input("Bitte 1. Zahl eingeben: "))
num_2 = float(input("Bitte 2. Zahl eingeben: "))

result = num_1 / num_2
print(f"Das Ergebnis der Division ist {result}")
\end{pythoncode}

Dieser Codeausschnitt ist problemlos ausführbar, solange der Nutzer für die Zahl \code{num\_2} eine Zahl ungleich 0 eingibt. Im Falle einer Division durch 0 wird ansonsten wie oben beschrieben ein \code{ZeroDivisionError} geworfen.\\~\

Diese Problematik muss der Programmierer bereits während der Entwicklung erkennen und eine entsprechende Fehlerbehandlung bereitstellen, falls der Nutzer tatsächlich die Zahl 0 eingeben sollte.
\end{exampleblock}    

    \end{frame}
    
    \begin{frame}[fragile,allowframebreaks]{Exception Handling (V)}
\begin{pythoncode}
num_1 = float(input("Bitte 1. Zahl eingeben: "))
num_2 = float(input("Bitte 2. Zahl eingeben: "))

try:
    result = num_1 / num_2
    print(f"Das Ergebnis der Division ist {result}")
except ZeroDivisionError:
    print("Diese Berechnung ist nicht möglich, als 2. Zahl muss ein Wert != 0 eingegeben werden.")
\end{pythoncode}

In diesem Fall ist der "problematische" Code (\code{result = num\_1 / num\_2}) in einem \code{try}-Block enthalten. Es wird also zunächst versucht, den Code fehlerfrei auszuführen. Ist dies möglich, wird der \code{except}-Block übersprungen. Dieser wird lediglich dann ausgeführt, wenn im \code{try}-Block die entsprechende Exception auftritt, die abgefangen werden soll (in diesem Fall \code{ZeroDivisionError}).\\  \framebreak

Im Beispiel wird also das Ergebnis der Kalkulation ausgegeben, falls \code{num\_2} != 0, andernfalls wird der Nutzer aufgefordert, als 2. Zahl einen anderen Wert als 0 einzugeben.\\~\

Dies bietet also den Vorteil, mögliche Fehler abzufangen und zu behandeln, anstatt das Programm zum Absturz zu führen.

    \end{frame}